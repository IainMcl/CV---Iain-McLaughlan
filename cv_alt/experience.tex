%-------------------------------------------------------------------------------
%	SECTION TITLE
%-------------------------------------------------------------------------------
\cvsection{Relevant Experience}

\begin{cvparagraph}
% Through learning to program at university in C and Python I have worked through several diffident projects both in my own time and as part of courses. Two of the more significant projects are detailed below. Both of these come with written reports which can be made available on request.
Whilst learning to program in C and Python I have worked through several different projects, in my own time, as university courses and while working for PHS. Some of the more significant projects are detailed below. Both of the university projects come with written reports which can be made available on request.
\end{cvparagraph}

\begin{cventries}
    \cventry
        {\href{www.sailingexercises.co.uk}{SailingExercises.co.uk}}
        {Sailing Exercises Website}
        {UK, Edinburgh}
        % {Feb 2020 - Present}
        {2020 - Present}
        {
        To help organise coaching ideas helping myself and other sailing coaches I developed a website using the Django python web framework. This site acts as a collaborative database to store and find different sailing exercises that can be searched for inspiration based on criteria such as time availability, number of participants, etc. The site has its back end developed in python with the front end built using vanilla Javascript, HTML5 and CSS3. The site is deployed using Digital Ocean running a PostgreSQL database. The site has various flourish add ons such as using crontab to post updates and most recent posts to Twitter daily.
        }
\end{cventries}

\begin{cventries}
    \cventry
        {University of Edinburgh}
        {Quantum Computing Project}
        {UK, Edinburgh}
        {Jan 2019 - May 2019}
        {
        % Developer role
        % Construction of a quantum computer simulator from scratch, which simulates the quantum phenomena found in quantum computers on a classical system using linear algebra to represent the effects. For collaborative work GitHub was used for version control, allowing for better work planning and time management.
        % Analyst Role
         Construction of a quantum computer simulator from scratch, which simulates the quantum phenomena found in quantum computers on a classical system using linear algebra to represent the effects. The performance of this was analyses through simulating many tests and statistical comparisons were made to expected results. For collaborative work \textit{GitHub} was used for version control, allowing for better work planning and time management.
        }
        
    \cventry
    {University of Edinburgh}
    {Estimating Galaxy Redshifts using Random Forests}
    {Uk, Edinburgh}
    {Jan 2019 - May 2019}
     {%Developer
     %As a final year project I compared different methods for estimating the redshifts of distant galaxies. This compared different machine learning methods such as \textit{Random Forests} and \textit{Self Organising Maps} along with more classic template fitting methods. The data was collected making use of \textit{SQL Databases} and analysed with python code making use of \textit{Numpy} and \textit{Pandas}. With the outputs being visualised with a mix of \textit{Matplotlib} and \textit{Gnuplot}.
     %analyst
     As a final year project I performed an analytic comparison of different methods for estimating the redshifts of distant galaxies. This compared different machine learning methods such as \textit{Random Forests} and \textit{Self Organising Maps} along with more classic template fitting methods. The data was collected making use of \textit{SQL Databases} and analysed with python code making use of \textit{Numpy} and \textit{Pandas}. With the outputs being visualised with a mix of \textit{Matplotlib} and \textit{Gnuplot}.}
    
 \cventry
    {Physics Undergraduate Student} % Job title
    {The University of Edinburgh} % Organization
    {UK, Edinburgh} % Location
    % {Sept. 2015 - Exp. May. 2019} % Date(s)
    {2015 - 2019} % Date(s)
    {
      While studying at the University of Edinburgh I have taken a range of courses resulting in the following skills: 
      \begin{itemize}
          \item Computing and Data Analysis.
          \item Problem Solving.
          \item Independent Learning
          \item Critical Thinking.
      \end{itemize}
      During my degree, I have been exposed to a wide range of experiences that have expanded my knowledge and skill base, including:
      \begin{itemize}
          \item Mathematical skills, from courses such as: Dynamics and Vector Calculus, Fourier Analysis and Statistics, Algebra and Calculus in preparation for courses in Quantum Physics and Electromagnetism.
          \item Being part of a project team to develop a quantum computer simulation in Python.
          \item Using \textit{ScikitLearn} to set up a machine learning weather prediction system.
          \item and one of my final year projects compared different methods that could be used to estimate the redshifts of galaxies that involved in-depth research to gain an understanding of the different methods; machine learning for some methods to gain a working understanding of the methods employed; data analysis to support quantitative and qualitative interrogation of the results and report writing.
      \end{itemize}
      }
    %   During my degree I have had a wide range of experiences. I have developed my mathematical skills by choosing maths for physics courses including Dynamics and Vector Calculus, Fourier Analysis and Statistics, Algebra and Calculus. This prepared me for courses such as Quantum Physics and Electromagnetism.} 
      
    %   In addition I have been involved with projects which cover more niche areas. I worked as part of a team to develop a quantum computer simulator using Python. I completed projects where $\it{Scikit\space Learn}$  was used to set up a machine learning weather prediction system. In my final year I undertook a project that made a comparison of different methods that could be used to estimate the redshifts of galaxies. This combined many skills in to one project. In depth research was required to gain an understanding of the different methods. Machine learning for some of the methods so a working understanding of the methods that were employed here were needed. Lastly, data analysis skills were used to quantitatively and qualitatively compare and understand the results of the project.
    % Maybe add some kind of concluding sentence about how uni has prepared me for stuff?}

%---------------------------------------------------------

\end{cventries}

% \newpage

\cvsection{Employment}



%-------------------------------------------------------------------------------
%	CONTENT
%-------------------------------------------------------------------------------
\begin{cventries}

%---------------------------------------------------------
    
% \cventry
%     {Information Analyst}
%     {Public Health Scotland (formerly National Services Scotland)}
%     {Uk, Edinburh}
%     {2019 - PRESENT}
%     {
%      Work as part of the Service Access team with three main roles: A&E, Drug and Alcohol Waiting Times and Systemwatch emergency hospital data. These roles produce regular publications for the public and media providing information about NHS performance. I regularly have to deal with information requests from both internal and external sources where I need to communicate with the client to produce the data that they need. An example of one of these was during the COVID-19 response in identifying where the reduction in A&E attendances were coming from. This work identified key health areas where people were not coming to hospital where they should. This work provided some of the stimulus and backing behind the government messages encouraging people to still attend A&E where needed and allowed them to target key health issues.
%      }
 \cventry
    {Information Analyst} % Job title
    {Public Health Scotland (formerly part of National Services Scotland)} % Organization
    {UK, Edinburgh} % Location
    {2019 - Present} % Date(s)
    {
    Work as part of the Service Access team with three main roles: A$\&$E, Drug and Alcohol Waiting Times and Systemwatch emergency hospital data. These roles produce regular publications for the public and media providing information about NHS performance. I regularly have to deal with information requests from both internal and external sources where I need to communicate with the client to produce the data that they need. An example of one of these was during the COVID-19 response in identifying where the reduction in A$\&$E attendances was coming from. This work identified key health areas where people were not coming to hospital where they should. This work provided some of the stimuli and backing behind the government messages encouraging people to attend A$\&$E where needed and allowed them to target key health issues.\newline
    Within this role, I have received several training opportunities including working as part of and leading an Agile team.
    }
    

%---------------------------------------------------------
  \cventry
    {Laser Performance Team Head Coach} % Job title
    {RYA Scotland} % Organization
    {UK, Scotland} % Location
    % {Jul. 2017 - Present} % Date(s)
    {2017 - Present} % Date(s)
    {
      Responsible for developing Scotland’s top youth (U19) Laser sailors to achieve their potential through leading regular training camps and providing coaching support at major events. 
      \begin{itemize}
          \item Identify key areas where performance can be improved using performance analysis and goal setting (and review).
          \item Develop a collaborative team working environment where sailors train and develop together by organising competitive games with a performance focus.
      \end{itemize}
      Key skills I demonstrate within this role include problem-solving, performance analysis, motivation, team development, teaching, coaching, time, risk and project management. Good organisation and clear goal setting with measurable, time-bound outcomes are essential within this position.
     }
    %   This involves identifying key areas where performance can be improved; finding solutions to challenging problems that may not have been seen before, and working with a range of personalities to bring everyone together as a team. Key skills I can demonstrate within this role include problem solving, motivation, team development, teaching, coaching, time, risk and project management. Clear goals with measurable, time bound outcomes are essential when delivering this role.}

%---------------------------------------------------------
  \cventry
    {Laser Class Academy Head Coach} % Job title
    {RYA Scotland} % Organization
    {UK, Scotland} % Location
    % {Jul. 2018 - PRESENT} % Date(s)
    {2018 - Present} % Date(s)
    {
    % Organise and deliver open training for Laser sailors whose abilities range from club to international level. 
    Responsible for:
    \begin{itemize}
        \item organising and developing open training for Laser sailors whose abilities range from club to international competitive level.
        \item leading a team of coaches to deliver an enjoyable and effective learning experience to sailors of all levels.
        % \item Deliver high level training to sailors with ranging experience from club to international level by leading a team of coaches to deliver a fun learning experience to sailors of all levels from club to international.
        % \item Adapt to individual sailors needs by identifying key areas of improvement and working towards an effective solution.
    \end{itemize}
    My coaching experience is useful when working with any team it had helped me understand the different attributes required to be effective.
    %   Within this position I am responsible for organising and delivering open training programmes that are available to all ages. I organize the coaching team to deliver a common goal irrespective of the sailors’ ability.  This adaptive style of coaching has translated into other areas of my life, when working with a team, as it has helped me to quickly identify people's areas of strength and weakness. This can be used to identify where people will be appropriately challenged so as to get the best from them. 
    }

%---------------------------------------------------------
 
  \cventry
    {Associate Instructor (Sailing)} % Job title
    {Sport Scotland, Cumbrae} % Organization
    {UK, Cumbrae (North Ayrshire)} % Location
    % {Apr. 2015 - PRESENT} % Date(s)
    {2015 - 2019} % Date(s)
    {
      As an associate instructor I provide a teaching role for Sport Scotland Cumbrae, where I work as part of a team with other instructors to deliver a range of RYA sailing courses from complete beginner to advanced racing. In this role, I often give people their first experiences trying a new activity where they need to learn a new and unfamiliar set of skills.
    }

%---------------------------------------------------------
 \cventry
    {Seasonal Dinghy Instructor (Sailing)} % Job title
    {Tighnabruaich Sailing School} % Organization
    {UK, Tighnabruaich} % Location
    {Jul. 2015} % Date(s)
    {
    }

%---------------------------------------------------------
\end{cventries}
